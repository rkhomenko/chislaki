\documentclass[a4paper,12pt]{article}
\usepackage[T2A]{fontenc}
\usepackage[utf8]{inputenc}
\usepackage[english,russian]{babel}
\usepackage{amsmath}
\usepackage{graphicx}
\usepackage{geometry}

\geometry{left=1cm}
\geometry{right=1cm}
\geometry{top=1cm}
\geometry{bottom=1.5cm}

\usepackage[normalem]{ulem}
\usepackage{hyperref}
\usepackage{listings}


\begin{document}

\begin{titlepage}
\begin{center}
    {Московский авиационный институт} \\
    {(национальный исследовательский университет)} \\
    {Кафедра 806}

\vspace{8cm}
\large{
    {Лабораторная работа №2} \\
    {по курсу <<Численные методы>>} \\
    {Тема: <<Решение нелинейных уравнений и систем нелинейных уравнений>>}
}
\end{center}

\vspace{6cm}
\begin{flushright}
\begin{minipage}{0.4\textwidth}
    \begin{flushleft}
        {Выполнил: студент группы 8О-308} \\
        {Хоменко Роман Дмитриевич} \\
        \vspace{0.5cm}
        {Преподаватель:} \\
        {к.ф.-м.н., доцент кафедры 806} \\
        {Иванов Игорь Эдуардович} \\
        \vspace{0.5cm}
        Оценка:
    \end{flushleft}
\end{minipage}
\end{flushright}

\vfill
\begin{center}
    {Москва, 2018}
\end{center}

\end{titlepage}

\section{Решение нелинейных уравнений}
\subsection{Задание}
Реализовать методы простой итерации и Ньютона решения нелинейных уравнений
в виде программ, задавая в качестве входных данных точность вычислений. С
использованием разработанного программного обеспечения найти положительный
корень нелинейного уравнения (начальное приближение определить графически).
Проанализировать зависимость погрешности вычислений от количества итераций.

\subsection{Метод простой итерации}
Будем решать трансцендентное уравнение вида
$$
f(x) = 0
$$


В начале использования любого метода решения трансцендентного уравнения
необходимо отделить корни, т.е. найти начальное приближение $x^{(0)}$
для корня. Сделать это можно графически.


Метод простой итерации предполагает представление уравнения в виде
$$
x = \phi(x).
$$
Тогда решение находится путем построения последовательности
$$
x^{(k + 1)} = \phi(x^{(k)}),\ k = 0, 1, 2, \ldots
$$

\textit{Условие сходимости итерационного процесса:} пусть функция $\phi(x)$ определена и дифференцируема на отрезке $[a, b]$. Тогда если
\begin{enumerate}
    \item $\phi(x) \in [a, b]\ \forall x \in [a, b],$
    \item $\exists q: |\phi'(x)| \leq q < 1\ \forall x \in (a, b),$
\end{enumerate}
то уравнение имеет и притом единственный корень $x^{*}$ на $[a, b]$.
К этому решению сходится итерационная последовательность $x^{(k)}$,
начинающаяся с любого $x^{(0)} \in [a, b]$. Для оценки погрешности
используются следующие формулы:
$$
|x^{(*)} - x^{(k + 1)} \leq \epsilon^{(k + 1)} =
    \frac{q}{1 - q} |x^{(k + 1)} - x^{(k)}|
$$

Следует отметить, что представление уравнения $f(x) = 0$ в виде $x = \phi(x)$
возможно далеко не всегда. В этом случае используют следующий прием. Будем
искать функцию $\phi(x)$ в виде
$$
\phi(x) = x + \lambda f(x).
$$
Используя условие $|\phi'(x)| < 1$, нетрудно получить следующую формулу:
$$
\lambda = \frac{sign f'(x)}{\underset{x \in [a, b]}{\max} |f'(x)|}.
$$

Для реализации универсальной процедуры решения требуется
написание процедуры дифференцирования. Это легко сделать, так как
$$
f'(x) \approx \frac{f(x + \Delta x) - f(x)}{\Delta x}.
$$
Таким образом, нам удалось полностью избавится от
каких-либо символьных вычислений для функции $f(x)$. Построение
функции $\phi(x)$ происходит численно.

\subsection{Метод Ньютона}
Для нахождения корня уравнения $f(x) = 0$ методом Ньютона, необходимо
построить следующий итерационный процесс
$$
x^{(k + 1)} = x^{(k)} - \frac{f(x^{(k)})}{f'(x^{(k)})},\ k = 0, 1, 2, \ldots
$$

\textit{Условие сходимости итерационного процесса:} пусть на отрезке
$[a, b]$ функция имеет первую и вторую производные постоянного знака и
пусть $f(a) f(b) < 0$. Тогда если $x^{(0)} \in [a, b]$ и
$$
f(x^{(0)}) f''(x^{(0)}) > 0,
$$
то начатая с нее последовательность $x^{(k)}$ cходится к корню
$x^{(*)} \in (a, b)$.

В качестве критерия окончания итерационного процесса используют
$$
|x^{(k + 1)} - x^{(k)}| < \epsilon
$$

Следует заметить, что если заменить $f'(x)$ на приближенное значение, то
получится следующая формула:
$$
x^{(k + 1)} = x^{(k)} - f(x^{(k)}) \frac{x^{(k)} - x^{(k - 1)}}{f(x^{(k)}) - f(x^{(k - 1)})}.
$$
Данный метод является двухшаговым, т.е. для вычисления результата $k + 1$-ой итерации,
необходимо знать значения на $k$-ой и $k - 1$-ой итерациях.

\subsection{Решение}
$$
f(x) = \ln{(x + 2)} - x^2 = 0
$$

\begin{figure}[htbp]
  \centering
  \includegraphics[width=0.8\textwidth]{figure1.png}
  \caption{Графики функций $\ln{(x + 2)}$ и $x^2$}
\end{figure}

\begin{verbatim}
$ ./chislaki-lab02-1 i 1e-20
************************ Fixed point iteration ***********************
x1 = -0.587609, f(x1) = -5.55112e-17
x2 = 1.0571, f(x2) = 0

$ ./chislaki-lab02-1 n 1e-20
**************************** Newton method ***************************
x1 = -0.587609, f(x1) = -5.55112e-17
x2 = 1.0571, f(x2) = 0
\end{verbatim}

\newpage

\section{Решение систем нелинейных уравнений}
\subsection{Задание}
Реализовать методы простой итерации и Ньютона решения систем нелинейных
уравнений в виде программного кода, задавая в качестве входных данных точность
вычислений. С использованием разработанного программного обеспечения решить
систему нелинейных уравнений (при наличии нескольких решений найти то из них, в
котором значения неизвестных являются положительными); начальное приближение
определить графически. Проанализировать зависимость погрешности вычислений от
количества итераций.

\subsection{Метод простой итерации}
Пусть дана система нелинейных уравнений c $n$ неизвестными
$$
\left\{
    \begin{array}{l}
        f_1(x_1, x_2, \ldots, x_n) = 0, \\
        f_2(x_1, x_2, \ldots, x_n) = 0, \\
        \vdots \\
        f_n(x_1, x_2, \ldots, x_n) = 0. \\
    \end{array}
    \right.
$$

Для ее решения методом простой итерации приведем
её к эквивалентной системе специального вида
$$
\left\{
    \begin{array}{l}
        x_1 = \phi_1(x_1, x_2, \ldots, x_n) = 0, \\
        x_2 = \phi_2(x_1, x_2, \ldots, x_n) = 0, \\
        \vdots \\
        x_n = \phi_n(x_1, x_2, \ldots, x_n) = 0, \\
    \end{array}
    \right.
$$
где функции $\phi_i$ - определенны и непрерывны в некотрой окрестности
искомого решения.


Если выбрано начальное приближение $x^{(0)}$, то последующие приближения
строятся в методе простой итерации находятся по формулам:
$$
\left\{
    \begin{array}{l}
        x_1^{(k +1)} = \phi_1(x_1^{(k)}, x_2^{(k)}, \ldots, x_n^{(k)}) = 0, \\
        x_2^{(k +1)} = \phi_2(x_1^{(k)}, x_2^{(k)}, \ldots, x_n^{(k)}) = 0, \\
        \vdots \\
        x_n^{(k +1)} = \phi_n(x_1^{(k)}, x_2^{(k)}, \ldots, x_n^{(k)}) = 0, \\
    \end{array}
    \right.
$$

\textit{Условие сходимости итерационного процесса:} пусть вектор функция
$\phi(x)$ непрерывная вместе со своей производной
$$
\phi'(x) =
\begin{pmatrix}
    \frac{\partial \phi_1(x)}{\partial x_1} & \frac{\partial \phi_1(x)}{\partial x_2}
    & \ldots & \frac{\partial \phi_1(x)}{\partial x_n} \\
    \frac{\partial \phi_2(x)}{\partial x_1} & \frac{\partial \phi_2(x)}{\partial x_2}
    & \ldots & \frac{\partial \phi_2(x)}{\partial x_n} \\
    \vdots & \vdots & \vdots & \vdots \\
    \frac{\partial \phi_n(x)}{\partial x_1} & \frac{\partial \phi_n(x)}{\partial x_2}
    & \ldots & \frac{\partial \phi_n(x)}{\partial x_n} \\
\end{pmatrix}
$$
в ограниченной выпуклой замкнутой области $G$ и
$$
\underset{x \in G}{\max} ||\phi'(x)|| \leq q < 1.
$$
Если $x^{(0)} in G$ и все последовательные приближения
$$
x^{(k + 1)} = \phi(x^{(k)}),\ k = 0, 1, 2, \ldots
$$
также содержатся в $G$, то итерационный процесс сходится
к единственному решению уравнения
$$
x = \phi(x)
$$
в области $G$ и справедливы следующие оценки погрешностей:
$$
||x^{(*)} - x^{(k + 1)}|| < \leq \frac{q^{k + 1}}{1 - q} ||x^{(1)} - x^{(0)}||,
$$
$$
||x^{(*)} - x^{(k + 1)}|| < \leq \frac{q}{1 - q} ||x^{(k + 1)} - x^{(k)}||.
$$

Как и в случае метода простой итерации для одномерного случае существует способ
избавится от символьных вычислений для приведения к эвкивалентоной
системе специального вида. Вектор функция $\phi(x)$ выбирается следующем образом:
$$
\phi(x) = x - J_f^{-1}(x^{(0)}) f(x),
$$
где $J_f$ - матрица Якоби, построенная по функциям $f_i$ исходной системы.
Критерием сходимости итерационного процесса при данном выборе $\phi(x)$
является
$$
||J_{\phi}(x^{(0)})|| = 0.
$$

\subsection{Метод Ньютона}
Итерационный процесс строится по формуле:
$$
x^{(k + 1)} = x^{(k)} + \Delta x^{(k)}.
$$

Найдем $\Delta x^{(k)}$
Можно разложить вектор функцию $f$ в ряд Тейлора в точке
с точностью до слагаемых порядка $(\Delta x^{(k)})^2$
$$
x^{(k + 1)} = x^{(k)} + \Delta x^{(k)}.
$$
Получим
$$
f(x^{(k + 1)}) = f(x^{(k)}) + J_f(x^{(k)}) \Delta x^{(k)} = 0.
$$

Выражая вектор приращений и подставляя его в формулу итерационного
процесса, получим
$$
x^{(k + 1)} = x^{(k)} - J_f^{-1}(x^{(k)}) f(x^{(k)}).
$$

\textit{Условие сходимости итерационного процесса:} пусть
функции $f_1$, $f_2$, \ldots, $f_n$ дифференцируемы в ограниченной
выпуклой замкнутой области $G$, и $|J_f(x^{(k)}| \not = 0$. Тогда
для $x^{(0)} \in G$ итерационный процесс сходится к точному решению
$x^{*} \in G$, причем сходимость квадратичная. В качестве условия окончания итерационного
процесса используется
$$
||x^{(k + 1)} -x^{(k)}|| \leq \epsilon.
$$

\subsection{Решение}
$$
\left\{
    \begin{array}{l}
        f_1(x) = (x_1^2 + a^2) x_2 - a^3 = 0, \\
        f_2(x) = (x_1 - \frac{a}{2})^2 + (x_2 - \frac{a}{2})^2 - a^ 2 = 0.
    \end{array}
    \right.
$$

$$
a = 3
$$

\begin{figure}[htbp]
  \centering
  \includegraphics[width=0.8\textwidth]{figure2.png}
  \caption{Графики функций $f_1$ и $f_2$}
\end{figure}

\begin{verbatim}
$ ./chislaki-lab02-2 i 1e-6
x1 =
      -1.324691
      2.5105056
f1(x1) = 1.3827332e-06
f2(x1) = 6.4515929e-07
x2 =
      4.4469471
     0.93830348
f1(x2) = 4.5008239e-08
f2(x2) = 8.6915666e-09

$ ./chislaki-lab02-2 n 1e-6
x1 =
      -1.324691
      2.5105056
f1(x1) = 1.3827332e-06
f2(x1) = 6.4515929e-07
x2 =
      4.4469471
     0.93830348
f1(x2) = 4.5008239e-08
f2(x2) = 8.6915666e-09
\end{verbatim}


\newpage

\section{Исходный код}
Исходный код доступен по ссылке
\url{https://github.com/rkhomenko/chislaki}.

\lstinputlisting[language=C++]{../../include/chislaki/matan/differentiation.hpp}
\lstinputlisting[language=C++]{../../include/chislaki/optimize/minimize.hpp}
\lstinputlisting[language=C++]{../../include/chislaki/transcendental/utility.hpp}
\lstinputlisting[language=C++]{../../tests/chislaki-lab02/src/chislaki-lab02-1.cpp}
\lstinputlisting[language=C++]{../../tests/chislaki-lab02/src/chislaki-lab02-2.cpp}

\end{document}
