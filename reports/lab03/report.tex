\documentclass[a4paper,12pt]{article}
\usepackage[T2A]{fontenc}
\usepackage[utf8]{inputenc}
\usepackage[english,russian]{babel}
\usepackage{amsmath}
\usepackage{graphicx}
\usepackage{geometry}

\geometry{left=1cm}
\geometry{right=1cm}
\geometry{top=1cm}
\geometry{bottom=1.5cm}

\usepackage[normalem]{ulem}
\usepackage{hyperref}
\usepackage{listings}


\begin{document}

\begin{titlepage}
\begin{center}
    {Московский авиационный институт} \\
    {(национальный исследовательский университет)} \\
    {Кафедра 806}

\vspace{8cm}
\large{
    {Лабораторная работа №3} \\
    {по курсу <<Численные методы>>} \\
    {Тема: <<Приближение функций. Численное дифференцирование и интегрирование>>}
}
\end{center}

\vspace{6cm}
\begin{flushright}
\begin{minipage}{0.4\textwidth}
    \begin{flushleft}
        {Выполнил: студент группы 8О-308} \\
        {Хоменко Роман Дмитриевич} \\
        \vspace{0.5cm}
        {Преподаватель:} \\
        {к.ф.-м.н., доцент кафедры 806} \\
        {Иванов Игорь Эдуардович} \\
        \vspace{0.5cm}
        Оценка:
    \end{flushleft}
\end{minipage}
\end{flushright}

\vfill
\begin{center}
    {Москва, 2018}
\end{center}

\end{titlepage}

\section{Интерполяция}
\subsection{Интерполяционные полиномы Лагранжа и Ньютона}
\subsubsection{Задание}
Используя таблицу значений $Y_i$ функции $y = f(x)$, вычисленных в точках
$X_i$, $i = 0, \ldots, 3$, построить интерполяционные многочлены Лагранжа
и Ньютона, проходящие через точки $\{X_i, Y_i\}$. Вычислить значение
погрешности интерполяции в точке $X^{*}$.

\subsubsection{Интерполяционный полином в форме Лагранжа}
Пусть на отрезке $[a, b]$ задано множество несовпадающих точек $x_i$,
в которых известны значения функции $f_i = f(x_i),\ i = 0, \ldots, n$.
Приближающая функция $\phi(x_i, a_0, \ldots, a_n)$, удовлетворяющая
условиям
$$
\phi(x_i, a_0, \ldots, a_n) = f(x_i) = f_i,\ i = 0, \ldots, n,
$$
называется интерполяционной.

К качестве интерполяционной функции часто берут полиномы $n$-ой степени:
$$
P_n(x) = \sum_{i = 0}^{\infty} a_i x^{i}.
$$

Подставляя значения узлов интерполяции и используя условие $P_n(x_i) = f_i$,
получаем СЛАУ относительно коэффициентов $a_i$:
$$
\sum_{i = 0}^{n} a_i x^{i} = f_k,\ k = 0, \ldots, n,
$$
которая, в случае несовпадения узлов интерполяции, имеет единственное решение.

Однако, можно не решать СЛАУ и записать многочлен в виде:
$$
L_n(x) = \sum_{i = 0}^{n} f_i l_i(x),
$$
где $l_i(x)$ удовлетворяет условию
$$
l_i(x_j) =
\left\{
    \begin{array}{l l }
        1, & i = j, \\
        0, & i \not = j.
    \end{array}
    \right.
$$

Очевидно, что данному условию будет удовлетворять
$$
l_i(x) = \prod_{j = 0, j \not = i}^{n} \frac{x - x_i}{x_i - x_j},
$$
а интерполяционный многочлен запишется в виде
$$
L_n(x) = \sum_{i = 0}^{n} f_i \prod_{j = 0, j \not = i} \frac{x - x_i}{x_i - x_j}.
$$

\subsubsection{Интерполяционный полином в форме Ньютона}
Для определения полинома в форме Ньютона введем понятие разделенной разности.
Разделенные разности нулевого порядка совпадают со значениями функции в узлах.
Разделенные разности первого порядка определяются через разделенные разности
нулевого порядка
$$
f(x_i, x_j) = \frac{f_i - f_j}{x_i - x_j}.
$$
Аналогично, разделенные разности третьего порядка выражаются через разности
второго порядка
$$
f(x_i, x_j, x_k) = \frac{f(x_i, x_j) - f(x_j, x_k)}{x_i - x_k}.
$$

Разделенная разность порядка $n - k + 2$ определяется как
$$
f(x_i, x_j, x_k, \ldots, x_{n - 1}, x_n) = \frac{f(x_i, x_j, x_k, \ldots, x_{n - 1}) - f(x_j, x_k, \ldots, x_n)}
{x_i - x_n}
$$

Запишем полином в форме Ньютона:
$$
P_n(x) = f(x_0) + (x - x_0) f(x_1, x_0) + (x - x_0)(x - x_1)f(x_0, x_1, x_2) + \ldots
+ (x - x_0)(x - x_1)\ldots(x- x_n)f(x_0, x_1, \ldots, x_n).
$$

Главное преимущество полинома в форме Ньютона в том, что при добавлении новой
точки не нужно пересчитывать все коэффициенты.

\subsubsection{Полиномы Чебышёва}
Одной из проблем при использовании полиномов в форме Лагранжа или Ньютона
является большая погрешность в точках, отличных от точек построения полинома.
Добавление большого числа дочек не <<сглаживает>> полином, а наоборот делает
его <<острым>>. Данной проблемы можно избежать, если строить полиномы Лагранжа
или Ньютона по корням полинома Чебышёва
$$
T_n(x) = \cos{(n \arccos(x))}
$$
$$
x_m = \cos{\frac{2m + 1}{2k}\pi},\ m = 0, \ldots, n - 1.
$$

\subsubsection{Решение}
$$
y = \cos{x}
$$
$$
X^{*} = \frac{\pi}{4}
$$

\begin{verbatim}
************************** Lagrange polynom **************************
y1:
              1
     0.92387953
     0.70710678
     0.38268343
y1 by lagrange:
              1
     0.92387953
     0.70710678
     0.38268343
y2:
              1
      0.8660254
     0.25881905
   6.123234e-17
y2 by lagrange:
              1
      0.8660254
     0.25881905
   6.123234e-17
y1 chebyshev:
              1
     0.92387953
     0.70710678
     0.38268343
y2 chebyshev:
              1
      0.8660254
     0.25881905
 -1.2223609e-08
f(x*) - L1(x*) = 0
f(x*) - L2(x*) = 0.0022963013
f(x*) - LChe(x*) = 2.220446e-16
*************************** Newton polynom ***************************
y1 by newton:
              1
     0.92387953
     0.70710678
     0.38268343
y2 by newton:
              1
      0.8660254
     0.25881905
   3.469447e-17
y1 chebyshev:
              1
     0.92387953
     0.70710678
     0.38268343
y2 chebyshev:
              1
      0.8660254
     0.25881905
   1.414155e-09
f(x*) - N1(x*) = 0
f(x*) - N2(x*) = 0.0022963013
f(x*) - NChe(x*) = 0
\end{verbatim}

\subsection{Сплайн-интерполяция}
\subsubsection{Задание}
Построить кубический сплайн для функции, заданной в узлах интерполяции,
предполагая, что сплайн имеет нулевую кривизну при $x = x_0$ и $x = x_4$.
Вычислить значение функции в точке $x = X^{*}$.

\subsubsection{Кубический сплайн}
Основная проблема интерполяции в помощью полиномов --
высокая степень полинома при большом числе точек. Для её решения
нужно отказаться от интерполяционной функции для всего диапазоне точек
и перейти к интерполяции кусочными функциями:
$S(x) = \sum_{k = 0}^{n} a_{ik} x^{k},\ x_{i - 1} \leq x \leq x_i,\ i = 1,\ldots, n,$

\end{document}
